% Options for packages loaded elsewhere
\PassOptionsToPackage{unicode}{hyperref}
\PassOptionsToPackage{hyphens}{url}
%
\documentclass[
  openany]{book}
\usepackage{amsmath,amssymb}
\usepackage{lmodern}
\usepackage{iftex}
\ifPDFTeX
  \usepackage[T1]{fontenc}
  \usepackage[utf8]{inputenc}
  \usepackage{textcomp} % provide euro and other symbols
\else % if luatex or xetex
  \usepackage{unicode-math}
  \defaultfontfeatures{Scale=MatchLowercase}
  \defaultfontfeatures[\rmfamily]{Ligatures=TeX,Scale=1}
\fi
% Use upquote if available, for straight quotes in verbatim environments
\IfFileExists{upquote.sty}{\usepackage{upquote}}{}
\IfFileExists{microtype.sty}{% use microtype if available
  \usepackage[]{microtype}
  \UseMicrotypeSet[protrusion]{basicmath} % disable protrusion for tt fonts
}{}
\makeatletter
\@ifundefined{KOMAClassName}{% if non-KOMA class
  \IfFileExists{parskip.sty}{%
    \usepackage{parskip}
  }{% else
    \setlength{\parindent}{0pt}
    \setlength{\parskip}{6pt plus 2pt minus 1pt}}
}{% if KOMA class
  \KOMAoptions{parskip=half}}
\makeatother
\usepackage{xcolor}
\IfFileExists{xurl.sty}{\usepackage{xurl}}{} % add URL line breaks if available
\IfFileExists{bookmark.sty}{\usepackage{bookmark}}{\usepackage{hyperref}}
\hypersetup{
  pdftitle={Documentation Technique MR-biosoft Annotation Platform},
  pdfauthor={Théo Roncalli; Gustavo Magaña López},
  hidelinks,
  pdfcreator={LaTeX via pandoc}}
\urlstyle{same} % disable monospaced font for URLs
\usepackage{color}
\usepackage{fancyvrb}
\newcommand{\VerbBar}{|}
\newcommand{\VERB}{\Verb[commandchars=\\\{\}]}
\DefineVerbatimEnvironment{Highlighting}{Verbatim}{commandchars=\\\{\}}
% Add ',fontsize=\small' for more characters per line
\usepackage{framed}
\definecolor{shadecolor}{RGB}{248,248,248}
\newenvironment{Shaded}{\begin{snugshade}}{\end{snugshade}}
\newcommand{\AlertTok}[1]{\textcolor[rgb]{0.94,0.16,0.16}{#1}}
\newcommand{\AnnotationTok}[1]{\textcolor[rgb]{0.56,0.35,0.01}{\textbf{\textit{#1}}}}
\newcommand{\AttributeTok}[1]{\textcolor[rgb]{0.77,0.63,0.00}{#1}}
\newcommand{\BaseNTok}[1]{\textcolor[rgb]{0.00,0.00,0.81}{#1}}
\newcommand{\BuiltInTok}[1]{#1}
\newcommand{\CharTok}[1]{\textcolor[rgb]{0.31,0.60,0.02}{#1}}
\newcommand{\CommentTok}[1]{\textcolor[rgb]{0.56,0.35,0.01}{\textit{#1}}}
\newcommand{\CommentVarTok}[1]{\textcolor[rgb]{0.56,0.35,0.01}{\textbf{\textit{#1}}}}
\newcommand{\ConstantTok}[1]{\textcolor[rgb]{0.00,0.00,0.00}{#1}}
\newcommand{\ControlFlowTok}[1]{\textcolor[rgb]{0.13,0.29,0.53}{\textbf{#1}}}
\newcommand{\DataTypeTok}[1]{\textcolor[rgb]{0.13,0.29,0.53}{#1}}
\newcommand{\DecValTok}[1]{\textcolor[rgb]{0.00,0.00,0.81}{#1}}
\newcommand{\DocumentationTok}[1]{\textcolor[rgb]{0.56,0.35,0.01}{\textbf{\textit{#1}}}}
\newcommand{\ErrorTok}[1]{\textcolor[rgb]{0.64,0.00,0.00}{\textbf{#1}}}
\newcommand{\ExtensionTok}[1]{#1}
\newcommand{\FloatTok}[1]{\textcolor[rgb]{0.00,0.00,0.81}{#1}}
\newcommand{\FunctionTok}[1]{\textcolor[rgb]{0.00,0.00,0.00}{#1}}
\newcommand{\ImportTok}[1]{#1}
\newcommand{\InformationTok}[1]{\textcolor[rgb]{0.56,0.35,0.01}{\textbf{\textit{#1}}}}
\newcommand{\KeywordTok}[1]{\textcolor[rgb]{0.13,0.29,0.53}{\textbf{#1}}}
\newcommand{\NormalTok}[1]{#1}
\newcommand{\OperatorTok}[1]{\textcolor[rgb]{0.81,0.36,0.00}{\textbf{#1}}}
\newcommand{\OtherTok}[1]{\textcolor[rgb]{0.56,0.35,0.01}{#1}}
\newcommand{\PreprocessorTok}[1]{\textcolor[rgb]{0.56,0.35,0.01}{\textit{#1}}}
\newcommand{\RegionMarkerTok}[1]{#1}
\newcommand{\SpecialCharTok}[1]{\textcolor[rgb]{0.00,0.00,0.00}{#1}}
\newcommand{\SpecialStringTok}[1]{\textcolor[rgb]{0.31,0.60,0.02}{#1}}
\newcommand{\StringTok}[1]{\textcolor[rgb]{0.31,0.60,0.02}{#1}}
\newcommand{\VariableTok}[1]{\textcolor[rgb]{0.00,0.00,0.00}{#1}}
\newcommand{\VerbatimStringTok}[1]{\textcolor[rgb]{0.31,0.60,0.02}{#1}}
\newcommand{\WarningTok}[1]{\textcolor[rgb]{0.56,0.35,0.01}{\textbf{\textit{#1}}}}
\usepackage{longtable,booktabs,array}
\usepackage{calc} % for calculating minipage widths
% Correct order of tables after \paragraph or \subparagraph
\usepackage{etoolbox}
\makeatletter
\patchcmd\longtable{\par}{\if@noskipsec\mbox{}\fi\par}{}{}
\makeatother
% Allow footnotes in longtable head/foot
\IfFileExists{footnotehyper.sty}{\usepackage{footnotehyper}}{\usepackage{footnote}}
\makesavenoteenv{longtable}
\usepackage{graphicx}
\makeatletter
\def\maxwidth{\ifdim\Gin@nat@width>\linewidth\linewidth\else\Gin@nat@width\fi}
\def\maxheight{\ifdim\Gin@nat@height>\textheight\textheight\else\Gin@nat@height\fi}
\makeatother
% Scale images if necessary, so that they will not overflow the page
% margins by default, and it is still possible to overwrite the defaults
% using explicit options in \includegraphics[width, height, ...]{}
\setkeys{Gin}{width=\maxwidth,height=\maxheight,keepaspectratio}
% Set default figure placement to htbp
\makeatletter
\def\fps@figure{htbp}
\makeatother
\setlength{\emergencystretch}{3em} % prevent overfull lines
\providecommand{\tightlist}{%
  \setlength{\itemsep}{0pt}\setlength{\parskip}{0pt}}
\setcounter{secnumdepth}{5}
\usepackage{booktabs}
\usepackage{amsthm}
\makeatletter
\def\thm@space@setup{%
  \thm@preskip=8pt plus 2pt minus 4pt
  \thm@postskip=\thm@preskip
}
\makeatother

\ifLuaTeX
  \usepackage{selnolig}  % disable illegal ligatures
\fi
\usepackage[]{natbib}
\bibliographystyle{apalike}

\title{Documentation Technique MR-biosoft Annotation Platform}
\author{Théo Roncalli \and Gustavo Magaña López}
\date{}

\begin{document}
\maketitle

{
\setcounter{tocdepth}{1}
\tableofcontents
}
\hypertarget{intro}{%
\chapter*{Introduction}\label{intro}}
\addcontentsline{toc}{chapter}{Introduction}

Le projet repose sur deux installations principalement: PostgreSQL et Python.
Le site a été conçu en utilisant \href{https://www.djangoproject.com/}{\emph{django}}, un
\emph{framework} pour le développement web écrit en langage Python. Toute la partie
\emph{front-end} du projet a été conçue dès zéro, en définissant les styles CSS
sans s'appuyer sur des \emph{frameworks} tels que bootstrap. Le système de gestion
de bases de données relationnelles choisi est \emph{PostgreSQL}.

Toutes les explications et instructions trouvées dans ce document
partent de l'hypothèse que \texttt{Python3.8+} et \texttt{psql\ (PostgreSQL)\ 12.7+} sont
installés dans votre système. Si vous n'avez pas PostgreSQL, je vous
recommande de suivre
\href{https://www.lri.fr/~schevalier/doc/teaching/tutoriels/PostgreSQL_install.html}{cette guide}.
Python est installé par défaut en la plupart
de distributions GNU/Linux. Tout le développement de l'application a été fait
en Ubuntu et Pop!\_OS, toutes les deux basées sur Debian. Aucune garantie n'est
donnée pour d'autres distributions.

\hypertarget{deps}{%
\section*{Gestion des Dépendances}\label{deps}}
\addcontentsline{toc}{section}{Gestion des Dépendances}

\begin{itemize}
\tightlist
\item
  \href{https://python-poetry.org/}{Poetry} for Python dependencies management.
\item
  \href{https://www.postgresql.org/}{PostgreSQL} for database management.
\end{itemize}

La gestion des dépendances du projet a été faite grâce à \emph{Poetry}, un gestionnaire
de paquets pour Python qui offre l'avantage d'être \(100\%\) reproductible,
contrairement à d'autres plateformes de gestion d'environnements très populaires
telles que \emph{conda}. Le manque de reproductibilité de conda vient du fait que son
\emph{channel} principal n'est pas un miroir de l'index de paquets python (\href{https://pypi.org/}{PyPI}).
Pour installer différents paquets, plusieurs \emph{channels} doivent être ajoutés.
Parfois on ne trouve pas le paquet et on finit par l'installer directement via \texttt{pip}.
Or, les paquets installés via \texttt{pip} ne seront pas suivis par conda. Si l'on essaie
de partager un fichier \texttt{env.yml} produit par conda, on n'a aucune garantie que tous
les paquets seront inclus dans le fichier.

Par contre, poetry a un \emph{dependency-solver} très robuste qui permet de trouver
des possibles conflits de dépendances lors de l'ajout via la commande \texttt{poetry\ add\ \textless{}dependency\textgreater{}}.
En étant stricte et informant l'utilisateur des conflits lors de l'ajout, en suggérant
comment les résoudre (si possible), poetry garantit la
Par défaut, poetry installe les paquets depuis le PyPI, ce qui veut dire qu'il peut
trouver tout paquet de l'index officiel de python.

Pour installer poetry et configurer les variables d'environnement nécessaires
afin de lancer le serveur, il est recommandé de suivre le guide
\href{https://github.com/MR-biosoft/AnnotationWebsite}{README du projet}.

Un script pour faciliter la configuration se trouve dans le livrable (pas dans
le repo car l'inclure dans un répertoire publique représenterait une grande
vulnerabilité).

Une fois que script \texttt{bootstrap\_installation.bash} exécuté, une nouvelle
session \texttt{bash} doit être démarrée afin d'avoir accès à la commande poetry.

À l'intérieur du répertoire où se trouvent \texttt{pyproject.toml} et \texttt{poetry.lock},
exécuter \texttt{poetry\ install} pour installer toutes les dépendances du projet.
Puis, exécuter \texttt{poetry\ shell} pour activer l'environnement virtuel et lancer
l'application.

\hypertarget{organisation}{%
\section*{Organisation du code serveur}\label{organisation}}
\addcontentsline{toc}{section}{Organisation du code serveur}

\begin{Shaded}
\begin{Highlighting}[]
\ExtensionTok{─}\NormalTok{ poetry.lock}
\ExtensionTok{─}\NormalTok{ pyproject.toml}
\ExtensionTok{─}\NormalTok{ prokaryote}
 \ExtensionTok{─}\NormalTok{ annotation }\CommentTok{\# Application principale}
 \ExtensionTok{─}\NormalTok{ bootstrap\_installation.bash }\CommentTok{\# script d\textquotesingle{}aide à l\textquotesingle{}installation}
 \ExtensionTok{─}\NormalTok{ createdb.bash }\CommentTok{\# idem mais pour la création des tables de la base de données}
 \ExtensionTok{─}\NormalTok{ dbsetup }\CommentTok{\# Application django pour gestion de la base de données}
 \ExtensionTok{─}\NormalTok{ debug.py }\CommentTok{\# Script pour analyser la sortie des error{-}logs }
 \ExtensionTok{─}\NormalTok{ gene\_importation\_error\_log.jsonl }\CommentTok{\# logs d\textquotesingle{}erreur liés à problèmes de parsing}
 \ExtensionTok{─}\NormalTok{ home }\CommentTok{\# Application qui gère la page d\textquotesingle{}accueil}
 \ExtensionTok{─}\NormalTok{ manage.py }\CommentTok{\# Couteau Suisse de django. }
 \ExtensionTok{─}\NormalTok{ prokaryote }\CommentTok{\# app. de base qui fait le lien entre toutes les autres}
 \ExtensionTok{─}\NormalTok{ runall.bash }\CommentTok{\# Script qui permet d\textquotesingle{}effacer et recréer la base et importer}
               \CommentTok{\# toutes les données}
 \ExtensionTok{─}\NormalTok{ static }\CommentTok{\# répertoire contenant le css de tout le site}
 \ExtensionTok{─}\NormalTok{ upload }\CommentTok{\# Application pour importer des nouvelles séquences}
\end{Highlighting}
\end{Shaded}

\hypertarget{django}{%
\chapter*{Introduction à django}\label{django}}
\addcontentsline{toc}{chapter}{Introduction à django}

django permet de construire des sites web modulaires, composés de plusieurs
applications indépendantes. Toute commande liée à un projet django passe par
le script \texttt{manage.py}. Celui-ci permet de lancer le serveur, se connecter à la
base de données de l'application, lancer une session Python interactive qui permet
de tester le code développé.

\hypertarget{commands}{%
\section*{Guide de commandes de base}\label{commands}}
\addcontentsline{toc}{section}{Guide de commandes de base}

\begin{itemize}
\tightlist
\item
  Créer un nouveau site : \texttt{django-admin\ startproject\ \textless{}project-name\textgreater{}}
\end{itemize}

À l'intérieur du dossier du projet, les commandes suivantes sont disponibles :

\begin{itemize}
\tightlist
\item
  Créer une nouvelle application : \texttt{./manage.py\ startapp\ \textless{}app-name\textgreater{}}
\item
  Se connecter à la base de données : \texttt{./manage.py\ dbshell}
\item
  Pour lancer une session python interactive : \texttt{./manage.py\ shell}
\end{itemize}

\hypertarget{commandsbis}{%
\section*{Commandes développées par l'équipe}\label{commandsbis}}
\addcontentsline{toc}{section}{Commandes développées par l'équipe}

\begin{itemize}
\tightlist
\item
  Exécuter un script sql à l'intérieur de la base de données du site :
  \texttt{./manage.py\ dbexec\ script.sql}
\item
  Importer des données (en format FASTA) :

  \begin{itemize}
  \tightlist
  \item
    Des génomes \texttt{./manage.py\ importgenome\ fichier\_genome.fa}
  \item
    Des gènes \texttt{./manage.py\ importgenes\ fichier\_genes.fa}
  \item
    Des protéines \texttt{./manage.py\ importproteins\ fichier\_proteines.fa}
  \end{itemize}
\end{itemize}

Le fichier \texttt{runall.bash} permet de tout exécuter: créer les tables de la base,
importer tous les génomes, tous les gènes et toutes les protéines (attention,
un dossier appelé \texttt{Data} doit contenir tous les fichier et doit se trouver
à la racine du projet).

  \bibliography{book.bib,packages.bib}

\end{document}
